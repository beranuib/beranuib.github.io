\begin{frame}
\end{frame}

\begin{frame}{Introducción}
\protect\hypertarget{introducciuxf3n}{}
IntroducciónIntroducción 1

Introducción:

\begin{itemize}
\item
  Compartir conocimiento como medio de avanzar.
\item
  Valorizar la cerveza, trago largo versus trago corto.
\item
  Seguiremos la línea básica de la elaboración en casa.
\item
  Buenas cervezas versus cervezas buenas.
\item
  Faig massa faltes d'ortografia si escric en català.
\end{itemize}
\end{frame}

\begin{frame}{Ingredientes}
\protect\hypertarget{ingredientes}{}
\begin{block}{Agua}
\protect\hypertarget{agua}{}
IngredientesAgua 1

AGUA\\
{image}

\begin{block}{Agua}
\protect\hypertarget{agua-1}{}
IngredientesAgua 2

\begin{itemize}
\item
  Los parámetros a tener en cuenta a la hora de elaborar una cerveza
  serán: Calcio, Magnesio, Sulfatos, Cloruro, Sodio y los Bicarbonatos
  que nos indicarán la "Dureza total".
\item
  También deberemos tener en cuenta el pH de nuestra agua y, si es el
  caso, intentar adaptarlo para obtener un pH adecuado en el macerado
  (5,3).
\item
  Para tratarla deberemos utilizar ácidos y sales, la cantidad de las
  cuales podremos obtener en calculadoras que encontraremos en Internet.
\end{itemize}

\begin{longtable}[]{@{}llllllll@{}}
\toprule
\textbf{CIUDAD} & \textbf{TIPO} & \textbf{Bicarbonato} & \textbf{Sodio}
& \textbf{Cloruro} & \textbf{Sulfatos} & \textbf{Calcio} &
\textbf{Magnesio} \\
\midrule
\endhead
BURTON & \emph{Pale Ale} & 200 & 24 & 36 & 800 & 294 & 24 \\
LONDRES & \emph{Mild/Bitter Ale} & 160 & 100 & 60 & 80 & 50 & 20 \\
PILSEN & \emph{Pilsen} & 15 & 2 & 5 & 5 & 7 & 2 \\
DUBLIN & \emph{Stout/Porter} & 200 & 12 & 19 & 55 & 115 & 4 \\
MUNICH & \emph{Lager alemana} & 180 & 2 & 60 & 120 & 75 & 18 \\
AMBERES & \emph{Ale belga} & 76 & 37 & 57 & 84 & 90 & 11 \\
\bottomrule
\end{longtable}
\end{block}
\end{block}

\begin{block}{Agua}
\protect\hypertarget{agua-2}{}
IngredientesAgua 3

DIFERENTES TIPOS DE AGUA\\
PARA DIFERENTES ESTILOS

{image}
\end{block}

\begin{block}{Agua}
\protect\hypertarget{agua-3}{}
IngredientesAgua 4

{image}
\end{block}

\begin{block}{Malta}
\protect\hypertarget{malta}{}
IntroducciónMalta 1

MALTA\\
{image}

\begin{block}{Malta2}
\protect\hypertarget{malta2}{}
IngredientesMalta 2 Normalmente utilizaremos cebada porque tiene un alto
contenido en enzimas que ayudarán a convertir las reservas energéticas
del grano en azúcares fermentables. Existen de 2,4 ó 6 carreras aunque
la más utilizada será la de dos carreras pues tiene menos proteínas y
produce más azúcares fermentables. El malteado del cereal pasa por las
siguientes etapas:\\

\begin{itemize}
\item
  \textbf{Limpieza:} El grano pasa por un proceso de filtrado que
  servirá para hacer una selección del mismo.
\item
  \textbf{Mojado:} Los granos se sumerjen en agua para que comiencen a
  germinar.
\item
  \textbf{Germinación:} El grano germina y se crean las enzimas, que
  destruyen los almidones, y las proteínas.
\item
  \textbf{Horneado:} Se seca en el horno y se cortan las raicillas. Se
  ha de dejar reposar un mes antes de utilizarlo.
\end{itemize}
\end{block}

\begin{block}{Malta3}
\protect\hypertarget{malta3}{}
IngredientesMalta 3

Maltas base:

\begin{itemize}
\item
  \textbf{Maltas base pálidas:} Pilsner, Pale Ale, Maris Otter.
\item
  \textbf{Maltas base tostadas más oscuras:} Munich o Vienna.
\item
  \textbf{Malta de trigo:} Produce azúcares fermentables y proteínas,
  las cuales ayudan con la espuma y producen turbidez.
\item
  \textbf{Malta de centeno:} Puede dar notas especiadas.
\end{itemize}

Maltas especiales:

\begin{itemize}
\item
  \textbf{Maltas caramelo:} También conocidas como maltas Crystal. Dan
  notas de miel, caramelo\ldots{}
\item
  \textbf{Maltas ámbar:} Sabor a galletas, añade un color ambar oscuro
  necesario en ciertos estilos.
\item
  \textbf{Malta tostadas:} Casi no tienen azucares, proporcionan
  colores, sabores y aromas complejos.
\end{itemize}
\end{block}

\begin{block}{Malta4}
\protect\hypertarget{malta4}{}
IngredientesColor El color de una cerveza vendrá dado, básicamente, por
el tipo de maltas utilizadas. En los editores de recetas nos indicará el
color aproximado que tendrá la cerveza una vez elaborada.

{image}
\end{block}
\end{block}

\begin{block}{Lúpulo}
\protect\hypertarget{luxfapulo}{}
IngredientesLúpulo 1

LÚPULO (Humulus lupulus)\\
{image}

\begin{block}{Lúpulo1}
\protect\hypertarget{luxfapulo1}{}
IngredientesLúpulo 2 El lúpulo es el encargado de dar el amargor y parte
de los sabores y aromas a la cerveza, asimismo la protege de
contaminaciones por parte de bacterias lácticas, acéticas u otros tipo
de microorganismo, al actuar como un antiséptico natural.\\
Las diferentes maneras en las que podemos utilizar el lúpulo son:\\

\begin{itemize}
\item
  \textbf{Flor:} La flor del lúpulo desecada.
\item
  \textbf{Pellet:} Lúpulo molido/machacado en polvo y luego capsulizado.
\item
  \textbf{Plugs:} Lúpulo desecado y comprimido en tabletas.
\item
  \textbf{Extractos de lúpulo:} Esencias o aceites aromáticos.
\end{itemize}

¿¿¿CÓMO CALCULAR LOS IBUS DE UNA RECETA???\\
\href{http://www.cerveceros-caseros.com/index.php/calculadora-acce/ibus}{{image}}\\
\end{block}

\begin{block}{Lúpulo2}
\protect\hypertarget{luxfapulo2}{}
IngredientesLúpulo 3

LÚPULOS

\begin{itemize}
\item
  \textbf{Lúpulos americanos} son lúpulos americanos de la era artesanal
  moderna con características cítricas, resinosas a hoja perennes o
  similares.
\item
  \textbf{Lúpulos del Viejo Mundo}, son los lúpulos del tipo Saaz pero
  también británicos y variedades de Europa continental. Tienen notas
  florales, especiadas, herbales o terrosas, generalmente menos intensas
  que los del Nuevo Mundo.
\item
  \textbf{Lúpulos del Nuevo Mundo}, provienen de Australia, Nueva
  Zelanda\ldots Tienen los mismos compuestos que los americanos, frutas
  tropicales, frutas de carozo, uva blanca, etc.
\item
  \textbf{Lúpulos tipo Saaz}, tambien llamados "Lúpulos Nobles", son los
  mejores lúpulos de la Europa Continental, con carácteres florales,
  especiados, o herbales, sin ser impetuosos o agresivos.
\end{itemize}
\end{block}
\end{block}

\begin{block}{Levaduras}
\protect\hypertarget{levaduras}{}
IngredientesLevadura 1

LEVADURA\\
{image}
\end{block}

\begin{block}{Levaduras}
\protect\hypertarget{levaduras-1}{}
IngredientesLevaduras 1 Las levaduras principalmente se dividiran en
\textbf{secas} o \textbf{líquidas}.

Levaduras secas

\begin{itemize}
\item
  Más fáciles de utilizar. Se trata de esterilizar, abrir el sobre,
  rehidratar si hace falta e inocular en el mosto.
\item
  Son más baratas que las levaduras líquidas.
\item
  Más fáciles de almacenar.
\end{itemize}

Levaduras líquidas

\begin{itemize}
\item
  El número de células vivas depende de la fecha de elaboración, por lo
  tanto puede que debamos elaborar un estarter.
\item
  Nos permite crear nuestro banco de levaduras e ir ampliándolo si es
  necesario.
\item
  Permiten más consistencia en las elaboraciones.
\end{itemize}

IngredientesLevaduras 2

TIPOS DE LEVADURAS:

SACCHAROMYCES

\begin{itemize}
\item
  Cerevisae, usada en cervezas del tipo Ale.
\item
  Bayanus, saccahromyces del sector del vino, cava, etc\ldots{}
\item
  Pastorianus, usada en cervezas del tipo Lagers.
\item
  Diastasicus, para cervezas tipo Saissons y demás cervezas que
  necesiten unas DF muy bajas. Se comen la maltotriosa.
\end{itemize}

NO-SACCHAROMYCES

\begin{itemize}
\item
  Brettanomyces. RIESGO DE CONTAMINACIÓN DE EQUIPOS. No hay riesgo para
  la salud, pero es aconsejable usar equipos exclusivos.
\item
  Bacterias: \textbf{lactobacillus}, bajan el pH y se consiguen en
  cervezas 'acidas y \textbf{pediococcus}, cervezas ácidas del tipo
  ``lámbico''
\item
  Levaduras salvajes o ``salvajes domesticadas'' como por ejemplo:
  Torulaspora delbrueckii, Metschnokowia pulcherrima\ldots{}
\end{itemize}
\end{block}
\end{frame}

\begin{frame}[fragile]{Procesos}
\protect\hypertarget{procesos}{}
\begin{block}{Procesos}
\protect\hypertarget{procesos-1}{}
ProcesosProcesos a seguir

= \[rectangle, draw, fill=brown!20, text width=9em, text centered,
rounded corners, minimum height=2em\] = \[draw, -latex'\]
\end{block}

\begin{block}{Elección del Estilo}
\protect\hypertarget{elecciuxf3n-del-estilo}{}
ProcesosElección del Estilo Diferentes formas de elaboración:

\begin{itemize}
\item
  \textbf{Kit:} Al kit se añade agua hirviendo, se refrigera y se añade
  la levadura.
\item
  \textbf{Extracto de malta:} Se añaden cereales, o no, se hierve y
  añade lúpulo, se refrigera y añadimos la levadura.
\item
  \textbf{Todo grano:} Se siguen los procesos que explicaremos\ldots{}
\end{itemize}

Elección del estilo Deberemos tener claro qué estilo deseamos elaborar,
teniendo claro que los ingredientes sean accesibles, que la receta esté
a la medida de nuestras capacidades y, sobre todo, que nos guste. Para
aprender más sobre los estilos es interesante conocer la guía de BJCP.\\
\href{https://www.thebeertimes.com/wp-content/uploads/2017/08/2015_Guidelines_Beer_Espa\%C3\%B1ol-final.pdf}{{image}}
\end{block}

\begin{block}{Compra de ingredientes}
\protect\hypertarget{compra-de-ingredientes}{}
ProcesosCompra de ingredientes

Para la compra de los ingredientes (o equipos) necesarios podremos
dirigirnos a:

\begin{itemize}
\item
  \textbf{Tiendas físicas}, podremos tener un asesoramiento aunque
  normalmente no disponen de tanta diversidad como las virtuales. Muchas
  también permiten compras por Internet.
\item
  \textbf{Tiendas virtuales} (en Internet), podremos escoger lo que
  necesitemos aunque, como en las tiendas físicas, no siempre podamos
  encontrar lo que buscamos en la misma tienda por lo que tendremos que
  recurrir a tablas de equivalencias (maltas, lúpulos o levaduras).

  \begin{itemize}
  \item
    \(\bullet\) \href{https://www.masmalta.com/es/}{Masmalta}
  \item
    \(\bullet\) \href{https://elsecretodelacerveza.com/}{El secreto de
    la cerveza}
  \item
    \(\bullet\) \href{https://www.cocinista.es/}{Cocinista}
  \item
    \(\bullet\) \href{https://hacercerveza.com/}{Hacer cerveza}
  \item
    \(\bullet\) \href{https://www.latiendadelcervecero.com/}{La tienda
    del cervecero}
  \end{itemize}
\end{itemize}
\end{block}

\begin{block}{Molienda}
\protect\hypertarget{molienda}{}
ProcesosMolienda El objetivo de moler la malta es prepararla para poder
conseguir la máxima obtención de azúcares y demás sustancias solubles.\\
Teóricamente el grano debería aplastarse en lugar de romperse, lo cual
es más fácil de conseguir con molinos de rodillos que con los del tipo
``corona''.\\
Es importante saber que podemos comprar la malta ya molida por un mínimo
incremento de precio, con lo cual podemos evitar comprar el molinillo,
pero entonces no será factible comprar sacos de 25 kg y moler la malta
el día de antes.
\end{block}

\begin{block}{Macerado}
\protect\hypertarget{macerado}{}
ProcesosMacerado 1 La maceración será la etapa en que introduciremos el
grano molido en agua caliente para que mediante la infusión se activen
las enzimas que sean capaces de convertir la reserva de energía del
grano, almidón, en azúcares fermentables.

Tipos de maceración

\begin{itemize}
\item
  \textbf{Infusión simple:} La temperatura se mantiene constante durante
  toda la maceración.
\item
  \textbf{Maceración escalonada:} Se inicia a una temperatura baja y se
  va incrementando. Aumentamos el rendimiento.
\item
  \textbf{Maceración por decocción:} Se mezcla entre 35 y 40ºC y se va
  aumentando, pero mientras tanto se va hirviendo una parte que
  posteriormente se incorpora al macerado. Pueden hacerse varios pasos.
\end{itemize}
\end{block}

\begin{block}{Macerado}
\protect\hypertarget{macerado-1}{}
ProcesosMacerado 2 El rango de temperaturas más habitual para macerar va
de 65 a 68ºC y la duración es de 1 hora.

Diferentes etapas o descansos en la maceración

\begin{itemize}
\item
  \textbf{Hidratación:} Entre 30 y 52ºC, servirá para bjar el pH de la
  maceración.
\item
  \textbf{Proteínica:} Sobre 45ºC, se forman aminoácidos y péptidos
  simples.
\item
  \textbf{Formación de azúcares:} Entre 55 y 65ºC, la \(\beta\)-amylasa
  crea azúcares más pequeños y muy fermentables.
\item
  \textbf{Formación de dextrinas:} Entre 67 y 72ºC, la
  \(\alpha\)-amylasa crea azúcares más grandes y no tan fermentables.
\item
  \textbf{\emph{Mash-out}:} Entre 74 y 77ºC, nos servirá para la
  desactivar las enzimas.
\end{itemize}
\end{block}

\begin{block}{Lavado}
\protect\hypertarget{lavado}{}
ProcesosLavado

El rango de temperaturas más habitual para lavar va de 74 a 77ºC y se ha
de hacer lentamente.

Tipos de lavado

\begin{itemize}
\item
  \textbf{Lavado continuo:} Es el método más utilizado y el más
  eficiente. Se rocía el grano con el agua de lavado.
\item
  \textbf{Lavado por etapas:} Se añade agua caliente al macerador, se
  remueve y esperamos 20 minutos para depués pasar al hervido.
\item
  \textbf{Metodo de NO lavado:} Se trasvasa directamente al hervidor sin
  hacer el lavado. Se pierden gran cantidad de azúcares.
\end{itemize}
\end{block}

\begin{block}{Hervido}
\protect\hypertarget{hervido}{}
ProcesosHervido

\begin{itemize}
\item
  El hervido debe ser lo más vigoroso posible y durar como mínimo
  sesenta minutos y no se debe tapar la olla para facilitar que
  desaparezcan los compuestos volátiles no deseados como el DMS (verdura
  cocida).
\item
  Antes de que llegue a hervir se puede añadir el lúpulo de primera
  adicción (First Wort Hopping). Serán los principales ``productores''
  del amargor que tenga esa cerveza.
\item
  Los lúpulos para dar sabor, sin que añadan mucho amargor, deberemos
  echarlos cuando falten veinte minutos para acabar el hervido.
\item
  Los lúpulos para dar aroma, y que no añadan amargor, deberemos
  echarlos cuando falten cinco minutos para acabar el hervido.
\item
  Diez minutos antes de acabar el hervido podemos añadir agentes
  clarificantes (Irish Moss) o nutrientes de levadura.
\end{itemize}
\end{block}

\begin{block}{Enfriado}
\protect\hypertarget{enfriado}{}
ProcesosEnfriado

Cuando haya acabado el hervido deberemos enfriar el mosto lo más rápido
posible para evitar contaminaciones.\\
Existen muchas maneras de enfriarlo\ldots pero las más eficientes son:

\begin{itemize}
\item
  \textbf{Serpentines:} Se ha de esterilizar previamente metiéndolo en
  la olla de hervido. Utilizad hielo en bloques para enfriar agua.
\item
  \textbf{Serpentines contracorriente:} Circula mosto caliente en un
  sentido y agua fría en otro.
\item
  \textbf{Enfriadores de placas:} Circula el mosto y el agua fría por
  una placas intercambiando calor.
\end{itemize}
\end{block}

\begin{block}{Fermentación}
\protect\hypertarget{fermentaciuxf3n}{}
ProcesosFermentación 1 La fermentación es el proceso más importante de
todos, es en él cuando se transformará el mosto que hemos elaborado en
cerveza.

\begin{itemize}
\item
  Deberemos inocular la cantidad adecuada de levadura, demasiada puede
  producir sabores no deseados y poca puede llevar a una fermentación
  incompleta.
\item
  Es importante oxigenar el mosto puesto que en la fase de reproducción
  éstas necesitarán oxigeno. Es el único momento en que se necesita, a
  posteriori siempre lo deberemos evitar para no oxidar la cerveza.
\end{itemize}

ACERCA DE LA DENSIDAD:

\begin{itemize}
\item
  \textbf{Grado de alcohol:} La diferencia de densidades nos indicará el
  grado de alcohol de la cerveza de forma aproximada.
\item
  \textbf{Densímetro:} Nos servirá para conocer la densidad del mosto.
  Existen otros instrumentos como el refractómetro, o refractómetros
  digitales.
\item
  \textbf{Controladores de fermentación:} A modo de curiosidad saber que
  existen dispositivos que se introducen en el fermentador y permiten
  seguir la fermentación a distancia. Por ejemplo: Tilt, iSpindel,
  etc\ldots{}
\end{itemize}
\end{block}

\begin{block}{Fermentación}
\protect\hypertarget{fermentaciuxf3n-1}{}
ProcesosFermentación 2

\begin{center}\rule{0.5\linewidth}{0.5pt}\end{center}

\begin{verbatim}
 El fermentador deberá disponer de un grifo y un borboteador (airlock).
 
 Necesitaremos conocer la densidad en varios momentos de la fementación.
\end{verbatim}

\begin{center}\rule{0.5\linewidth}{0.5pt}\end{center}
\end{block}

\begin{block}{Trasiego}
\protect\hypertarget{trasiego}{}
ProcesosTrasiego

Cuando la fermentación ha acabado (han transcurrido dos días sin que
varíe la densidad en el densímetro) podremos decir que el mosto se ha
convertido en cerveza. Será el momento de sacarla del fermentador e
introducirla en otro recipiente más adecuado. Básicamente puede tratar
de\ldots{}\\

\begin{itemize}
\item
  \textbf{Botellas}, siempre oscuras para evitar ``golpes de luz'',
  sabores y olores azufrados, a ``pedo de mofeta''.
\item
  \textbf{Barriles}, los típicos que encontraremos en un bar asociados a
  un tirador y una botella de \(CO_2\) para empujar la cerveza.
\item
  \textbf{Cornis}, ideales los de 9 l, los más comunes son de 19 l.
  Necesitan disponer de \(CO_2\) y un grifo.
\end{itemize}
\end{block}

\begin{block}{Carbonatación}
\protect\hypertarget{carbonataciuxf3n}{}
ProcesosCarbonatación Una vez embotellada la cerveza, ésta deberá
fermentar por segunda vez pero ya dentro de la botella y sin posibilidad
de que escape el \(CO_2\) que se producirá en la fermentación
secundaria.\\
Se puede utilizar para esta fermentación: Dextrosa, Azúcar o Extracto de
Malta.\\
Hay que tener cuidado, una sobrecarbonatación puede hacer que las
botellas exploten. No es un mito.\\
Para calcular la cantidad de ``azúcar'' que hay que añadir utilizaremos
una calculadora de Internet, por ejemplo\ldots{}\\
\href{http://www.cerveceros-caseros.com/index.php/calculadora-acce/carbonatacion}{{image}}\\
Existen diferentes métodos para añadir azúcares al mosto, se puede meter
todo en el fermentador o individualmente en cada botella.\\
Los métodos de carbonatar barriles o cornis son diferentes\ldots{}
\end{block}

\begin{block}{Carbonatación}
\protect\hypertarget{carbonataciuxf3n-1}{}
ProcesosPráctica de carbonatación Ejemplo de cómo podemos llevar a la
práctica el proceso.\\
Suponemos los siguientes datos: {image}\\
En un vaso de precipitación ponemos los gramos de dextrosa indicado y
completamos hasta 500 ml de agua. Disolvemos la dextrosa y la llevamos a
hervir unos dos o tres minutos. Mientras lavamos las botellas y las
ponemos a escurrir.\\

\[\binom{151 g.l \longrightarrow 500 ml}{7,6 g  \longrightarrow  x} \Rightarrow x= \frac{500 ml * 7,6 \cancel{g}}{151\cancel{g}l} = 25,16 \frac{ml}{l}\]

Por lo que si estamos embotellando en botellas de 33cl, deberemos
introducir en cada una de ellas:\\

\[\frac{25,16 ml}{3} = 8,4 ml \hspace{1cm}\]

Llenamos la botella, agitamos y chapamos.
\end{block}

\begin{block}{Chapado}
\protect\hypertarget{chapado}{}
ProcesosChapado

\begin{center}\rule{0.5\linewidth}{0.5pt}\end{center}

\begin{verbatim}
 Para cerrar las botellas necesitaremos una chapadora y chapas, que pueden ser de 26mm o de 29mm si se trata de botellas de cava.
 
 Es muy importante la limpieza de las botellas, para ello se pueden utilizar dispositivos específicos.
\end{verbatim}

\begin{center}\rule{0.5\linewidth}{0.5pt}\end{center}
\end{block}

\begin{block}{Etiquetado}
\protect\hypertarget{etiquetado}{}
ProcesosEtiquetado Debemos tener en cuenta que las etiquetas se deberán
imprimir SIEMPRE en una impresora láser (no de chorro de tinta) y para
pegarlas a la botella podemos utilizar leche entera.\\
{image}
\end{block}

\begin{block}{Cata}
\protect\hypertarget{cata}{}
ProcesosCata

Lo ideal es seguir una hoja de cata de las existentes en Internet. Mi
recomendación sería cumplimentar la del BJCP.\\
\href{https://bjcp.org/intl/Exam-Scoresheet-ES.pdf}{{image}}
\end{block}
\end{frame}

\begin{frame}{Apéndices}
\protect\hypertarget{apuxe9ndices}{}
\begin{block}{Higiene}
\protect\hypertarget{higiene}{}
ApéndiceHigiene Primero, y antes que nada, deberemos limpiar con agua y
jabón para depués desinfectar. Para ello tendremos:

Star San (Chemipro San):

\begin{itemize}
\item
  Mayormente es ácido fosfórico, también utilizado para bajar el pH.
\item
  Se debe dejar actuar durante un minuto.
\item
  Debemos diluirlo tal como indica el fabricante
\item
  No es necesario enjuagar.
\end{itemize}

Chemipro:

\begin{itemize}
\item
  Se trata de percarbonato sódico.
\item
  Una cucharada de café por cada litro de agua caliente.
\item
  Debe actuar entre 2 y 5 minutos.
\item
  Si se escurre bien no es necesario enjuagar.
\end{itemize}
\end{block}

\begin{block}{Defectos en la cerveza}
\protect\hypertarget{defectos-en-la-cerveza}{}
ApéndiceDefectos en la cerveza

2 \textbf{Acetaldehído}: Manzanas verdes frescas recién cortadas.

\begin{itemize}
\item
  Asegurar una fermentación vigorosa.
\item
  Permitir una atenuación completa.
\item
  Oxigenar el mosto plenamente.
\end{itemize}

\textbf{Alcohólico / Caliente}: Especiado, vínoso, calidez por etanol y
alcoholes superiores.

\begin{itemize}
\item
  Reduzca la temperatura de fermentación.
\item
  Utilice una cepa de levadura menos atenuante.
\item
  Compruebe la salud de la levadura.
\end{itemize}

\textbf{Astringente:} Fruncimiento de labios, aspereza persistente en
lengua.

\begin{itemize}
\item
  No sobrelavar.
\item
  No moler demasiado el grano.
\item
  No lavar con agua con un PH alto (sobre 6).
\end{itemize}

\textbf{Diacetil}: Mantequilla, caramelo, palomitas de maíz (pop corn).

\begin{itemize}
\item
  Oxigene el mosto antes de la fermentación.
\item
  Use levadura saludable en cantidad suficiente.
\item
  Si es una cerveza lager, elevar la temperatura para reposar el
  diacetilo al final de la fermentación.
\end{itemize}

\textbf{DMS (Dimetil Sulfuro)}: Maíz cocido.

\begin{itemize}
\item
  Utilice un hervido largo, continuo y abierto.
\item
  Reducir la cantidad de malta Pilsner.
\item
  Enfriar rápidamente el mosto antes de lanzar la levadura.
\end{itemize}

\textbf{Ésteres}: Frutal (fresa, pera, plátano, manzana, uva, cítricos).

\begin{itemize}
\item
  Reduzca la temperatura de fermentación.
\item
  Oxigenar suficientemente el mosto.
\item
  Compruebe si la variedad del lúpulo tiene características frutales.
\end{itemize}

\textbf{Exposición a la luz}: Olor a zorrillo, a orina de gato.

\begin{itemize}
\item
  No exponga el mosto/cerveza a la luz del sol después de añadir el
  lúpulo.
\item
  No utilice botellas claras o de vidrio verde.
\item
  Evitar el uso de lúpulo en cogollo en las adiciones finales.
\end{itemize}

\textbf{Oxidado}: Rancio, a papel, a cartón.

\begin{itemize}
\item
  No salpique cuando trasvasije y embotelle..
\item
  Purifique botellas/barriles con \(CO_{2}\) antes del llenado.
\item
  Consuma la cerveza cuando está fresca.
\end{itemize}

\textbf{Fenólico}: Clavo, pimienta, vainilla, etc.

\begin{itemize}
\item
  Use una cepa diferente de levadura y/o variedad de lúpulo.
\item
  Ajuste la T° de fermentación (mayor o menor dependiendo de la cepa de
  levadura y el estilo de cerveza).
\end{itemize}

\textbf{Sulfuro (Azufre)}: Huevos podridos, fósforos quemados, caucho
quemado.

\begin{itemize}
\item
  Verifique infecciones.
\item
  Controle el agua para sulfatos excesivos.
\item
  Verifique la salud de la levadura.
\end{itemize}

\textbf{Vinagre}: Ácido acético, acidez similar al vinagre.

\begin{itemize}
\item
  Verifique infecciones.
\item
  Compruebe la cepa de levadura.
\item
  Verifique fuentes de oxidación (las aceto-bacterias son aeróbicas)
\end{itemize}

2
\end{block}
\end{frame}

\begin{frame}{Información adicional}
\protect\hypertarget{informaciuxf3n-adicional}{}
\begin{block}{Enlaces interesantes en castellano}
\protect\hypertarget{enlaces-interesantes-en-castellano}{}
Información adicionalEnlaces interesantes en castellano

ENLACES INTERESANTES CASTELLANO.

\begin{itemize}
\item
  \url{http://www.cerveceros-caseros.com/}
\item
  \url{https://homebrewer.es/}
\item
  Guía BJCP
  \url{https://www.thebeertimes.com/wp-content/uploads/2017/08/2015_Guidelines_Beer_Espa\%C3\%B1ol-final.pdf}
\item
  \url{https://birrocracia.wordpress.com/}
\item
  \url{https://cervezomicon.com/}
\item
  Tabla de lúpulos
  \url{https://www.cocinista.es/web/es/recetas/hacer-cerveza/trucos-y-consejos/caracteristicas-de-los-lupulos-y-sus-sustitutos.html}
\end{itemize}
\end{block}
\end{frame}

\begin{frame}{Información adicional}
\protect\hypertarget{informaciuxf3n-adicional-1}{}
\begin{block}{Enlaces interesantes en inglés}
\protect\hypertarget{enlaces-interesantes-en-ingluxe9s}{}
Información adicionalEnlaces interesantes en inglés \uline{ENLACES
INTERESANTES INGLÉS}\\

\begin{itemize}
\item
  \url{https://community.grainfather.com/}\strut \\
\item
  \url{https://www.brewersfriend.com/}\strut \\
\item
  \url{http://www.milkthefunk.com/}\strut \\
\end{itemize}
\end{block}
\end{frame}

\begin{frame}{Información adicional}
\protect\hypertarget{informaciuxf3n-adicional-2}{}
\begin{block}{Enlaces interesantes en catalán}
\protect\hypertarget{enlaces-interesantes-en-cataluxe1n}{}
Información adicionalEnlaces interesantes en catalán \uline{ENLACES
INTERESANTES EN CATALÁN}\\

\begin{itemize}
\item
  \url{http://eltiradordecervesa.blogspot.com/}\strut \\
\item
  \url{http://cervesaencatala.blogspot.com/}\strut \\
\end{itemize}
\end{block}
\end{frame}
